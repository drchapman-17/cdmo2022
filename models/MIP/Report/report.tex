%% If writing in English, remove the lines above
%% and uncomment the lines below

% ------------- English version ---------------
\documentclass[english]{sbrt}
\usepackage[english]{babel}
\usepackage[utf8]{inputenc}
\newtheorem{theorem}{Theorem}
% ---------------------------------------------

\begin{document}

\title{VLSI Problem: MILP Formulation}

\author{Scroto 
\thanks{ZIOPINO }%
}

\maketitle

\markboth{BOH}{}

\begin{abstract}
In this paper, we consider a floorplanning problem
in the physical design of very large scale integration. We focus
on the problem of placing a set of blocks (modules) on a chip
with the objective of minimizing area of the chip. We will propose a method based on a linear programming and simulated annealing (SBORRA).  
\end{abstract}
\begin{keywords}
Floorplanning, Linear Programming, Simulated Annealing
\end{keywords}


\section{Introduction}

Sì

%%______________MODEL FORMULATION______________%%
\section{Model Formulation}

\subsection{Parameters}
\begin{enumerate}
\item[$\eta$] Number of blocks
\item[$w_c$] Width of the chip
\item[$p_i$] Length of the shorter side of the block $i$
\item[$q_i$] Length of the longer side of the block $i$
\item[$M$] A very large positive number
\end{enumerate}

\subsection{Decision Variables}
\begin{enumerate}
\item[$w_i,h_i$] Width and height of block $i$
\item[$x_i^l,x_i^r$] $x$ coordinates of the left and right boundaries of block $i$
\item[$y_i^b,y_i^t$] $x$ coordinates of the bottom and top boundaries of block $i$
\item[$h_c$] Height of the chip
\item[$r_{ij}$] =0 if block $i$ is to be placed to the left of block $j$, 1 otherwise (i.e, block $i$ is free to be placed on any side of block $j$)
\item[$r_{ij}$] =0 if block $i$ is to be placed below block $j$, 1 otherwise
\item[$v_{ij}$] =0 if block $i$ is placed horizontally, 1 otherwise (i.e, when the block is rotated)
\end{enumerate}

\subsection{Equations}
$$\text{Minimize} h_c}$
subject to
\begin{align}
x_i^r-x_i^l&=w_i \forall i
x_i^t-x_i^b&=w_i \forall i
w_i=v_i\cdot p_i+(1-v_i)

\end{align}

\section{Experimental results}
La sburra

\begin{thebibliography}{99}
\bibitem{ref1} L. Lamport, \textit{A Document Preparation System: \LaTeX, User's
Guide and Reference Manual}. Addison Wesley Publishing Company,
1986.
\bibitem{ref2} F. C. Silva e J. J. Sousa, ``Esta referência é apenas um exemplo," ~\textit{Revista de Exemplos}, v. 5, pp. 52--55, Maio
1999.
\end{thebibliography}


\appendix
Inserir as informações referentes aos apêndices aqui.


\end{document}
